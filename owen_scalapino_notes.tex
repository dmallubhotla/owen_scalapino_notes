\documentclass{article}

%other packages
\usepackage{amsmath}
\usepackage{amssymb}
\usepackage{physics}

\usepackage[
	style=phys, articletitle=false, biblabel=brackets, chaptertitle=false, pageranges=false, url=true
]{biblatex}

\usepackage{graphicx}
\usepackage{todonotes}
\usepackage{siunitx}

\usepackage{cleveref}

\title{Notes on Owen-Scalapino}

\addbibresource{./bibliography.bib}

\graphicspath{{./figures/}}

\newcommand{\pf}{p_{\mathrm{F}}}
\newcommand{\vf}{v_{\mathrm{F}}}
\newcommand{\corr}{\mu^\ast}
\DeclareMathOperator{\sgn}{sgn}

\begin{document}

\maketitle

Let's look at how we can translate the Owen-Scalapino results (OS)\cite{OwenScalapino} into something usable for our purposes for modelling the non-equilibrium conductivity.

\section{Main results from OS} \label{sec:intro}

The main results from OS 

\printbibliography

\end{document}
