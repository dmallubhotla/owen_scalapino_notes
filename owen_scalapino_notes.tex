\documentclass{article}

%other packages
\usepackage{amsmath}
\usepackage{amssymb}
\usepackage{physics}

\usepackage[
	style=phys, articletitle=false, biblabel=brackets, chaptertitle=false, pageranges=false, url=true
]{biblatex}

\usepackage{graphicx}
\usepackage{todonotes}
\usepackage{siunitx}

\usepackage{cleveref}

\title{Notes on Owen-Scalapino}

\addbibresource{./bibliography.bib}

\graphicspath{{./figures/}}

\newcommand{\pf}{p_{\mathrm{F}}}
\newcommand{\vf}{v_{\mathrm{F}}}
\newcommand{\corr}{\mu}
\newcommand{\step}{\Theta}
\DeclareMathOperator{\sgn}{sgn}

\begin{document}

\maketitle

Let's look at how we can translate the Owen-Scalapino results (OS)\cite{OwenScalapino} into something usable for our purposes for modelling the non-equilibrium conductivity.

\section{Main results from OS} \label{sec:intro}

The main expressions of interest in the OS paper are their equations 6 and 7.
The former is the modification to the gap equation with the ``chemical potential'' $\corr$:
\begin{equation}
	\left[ N(0) V \right]^{-1} = \int_{- \omega_D}^{\omega_D} \frac{\dd{\epsilon_k}}{\sqrt{\Delta^2 + \epsilon_k^2}} \tanh{\frac{\sqrt{\Delta^2 + \epsilon_k^2} - \corr}{2 T}}. \label{eq:gap}
\end{equation}
OS's equation 7 is the definition of a parameter $n$ measuring the excess quasiparticle density:
\begin{equation}
	n = \frac{1}{\Delta_0} \int_0^\infty \dd{\epsilon} \left( \frac{1}{1 + \exp(\frac{\sqrt{\Delta^2 + \epsilon_k^2} - \corr}{T})} - \frac{1}{1 + \exp(\frac{\sqrt{\Delta^2 + \epsilon_k^2}}{T})} \right). \label{eq:n}
\end{equation}

\subsection{Zero-temp limits} \label{subsec:zerotemp}
It's useful to begin with the zero temperature limit of \cref{eq:n}.
Without too much work, it's easy to see that
\begin{equation}
	\lim_{T\rightarrow 0} \frac{1}{1 + \exp\frac{A}{T}} = \begin{cases}
		0 & A > 0 \\
		1 & A < 0
	\end{cases},
\end{equation}
or alternatively using the Heaviside step function $\step$,
\begin{equation}
	\lim_{T\rightarrow 0} \frac{1}{1 + \exp\frac{A}{T}} = \step(- A).
\end{equation}
This means that \cref{eq:n} can, in the zero temp limit, be written
\begin{align}
	n(T = 0) = \frac{1}{\Delta_0} \int_0^\infty \dd{\epsilon} \left(\step\left(\corr - \sqrt{\Delta^2 + \epsilon_k^2}\right) - \step\left(- \sqrt{\Delta^2 + \epsilon_k^2}\right) \right).
\end{align}
The second step function is always negative, so
\begin{align}
	n(T = 0) &= \frac{1}{\Delta_0} \int_0^\infty \dd{\epsilon} \step\left(\corr - \sqrt{\Delta^2 + \epsilon_k^2}\right) \\
	&= \frac{1}{\Delta_0} \int_0^{\epsilon^\ast} \dd{\epsilon}  \\
	&= \frac{\epsilon^\ast}{\Delta_0},
\end{align}
where $\epsilon^\ast$ is the value of $\epsilon$ where the step function becomes zero, so
\begin{equation}
	{\epsilon^\ast}^2 = \corr^2 - \Delta^2,
\end{equation}
meaning that
\begin{equation}
	n = \frac{\sqrt{\corr^2 - \Delta^2}}{\Delta_0}, \label{eq:os9}
\end{equation}
which is simply equation 9 in OS.

We can look closer at OS's equation 8, the zero temp limit of \cref{eq:gap}:
\begin{equation}
	\frac{\Delta^3}{\Delta_0^3} = \left(\left(\frac{\Delta^2}{\Delta_0^2} + n^2\right)^{\frac12} - n \right)^2
\end{equation}
If we insert \cref{eq:os9}, we get
\begin{align}
	\frac{\Delta^3}{\Delta_0^3} &= \left(\left(\frac{\Delta^2}{\Delta_0^2} + \frac{\corr^2 - \Delta^2}{\Delta_0^2}\right)^{\frac12} - n \right)^2 \\
	\frac{\Delta^3}{\Delta_0^3} &= \left(\left(\frac{\corr^2}{\Delta_0^2}\right)^{\frac12} - n \right)^2 \\
	\frac{\Delta^3}{\Delta_0^3} &= \left(\frac{\corr}{\Delta_0} - n \right)^2 \\
	\Delta^3 &= \Delta_0 \left(\corr - n \Delta_0 \right)^2 \\
	\Delta^3 &= \Delta_0 \left(\corr - \sqrt{\corr^2 - \Delta^2}\right)^2, \label{eq:os8mu}
\end{align}
which should be a zero temp approximation of the modified gap equation without reference to $n$.

However, this is a problematic equation, and numerical solutions in this limiting case do not actually satisfy \cref{eq:gap}, at least for the values I've tried.
For instance, start with $\omega_D = 100$ and $\left[ N(0) V \right]^{-1} = 5$, leading to $\Delta_0 \approx 1.35$ (with those values specifically chosen to have a gap around order 1).
Then, solve \cref{eq:os8mu} numerically with $\corr = 1.5$, which is a valid value that should lead to some nonzero $n$, because it is greater than $\Delta_0$.
This gives $\Delta \approx 0.46$, in addition to the trivial $\Delta = 0$ solutions.
This is not close to satisfying \cref{eq:gap}.

I think this is likely a problem with that particular zero temperature limit, which I don't have a ton of confidence in.

It's worth noting as well, the integral can also be calculated using a similar process as we used for the $n$ integral, giving us
\begin{equation}
	\frac{\Delta}{\Delta_0} = \frac{\min(\omega_D, \sqrt{\corr^2 - \Delta^2})}{\omega_D}
\end{equation}
as another condition at zero temperature.

\section{Mathematica calculation issues}

In any case, we care more about non-zero $T$, so we wil need to use the integral equations directly.
I think there's still freedom to choose whether or not we take $\corr$ or $n$ as the independent variable;
This point is still not yet clear to me, although it seems like $n$ would be slightly easier to measure in an independent way (either by looking at some current being ejected by the SC or something similar, which should be proportional to $n$?).\todo{come back to this point}
Either way, as mentioned earlier the interesting calculations are for fixed quasiparticle concentration.

On the object level, this corresponds to solving the simultaneous integral equations \cref{eq:gap} and \cref{eq:n}, which Mathematica has a little bit of trouble on.
Obtaining functional code for this is the next order of business.

\printbibliography

\end{document}
